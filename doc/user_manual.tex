
%%%%%%%%%%%%%%%%%%%%%%%%%%%%%%%%%%%%%%%%%%%%%%%%%%%%%%%%%%%%%%%%%%%%%
% FOR PRINTING
\documentclass[
    11pt,
    a4paper,
    openright,
    cleardoubleplain,
    appendixprefix,
    chapterprefix,
    numbers=noenddot,
    DIV11,
    BCOR15mm
]{report}
               
\newcommand{\eztrace}{EZTrace}
\newcommand{\reporttitle}{\eztrace}               
\newcommand{\reportsubtitle}{A Generic Framework for\\ Instrumenting Applications}
\newcommand{\reportsubsubtitle}{User Manual}
%%%%%%%%%%%%%%%%%%%%%%%%%%%%%%%%%%%%%%%%%%%%%%%%%%%%%%%%%%%%%%%%%%%%%
\usepackage{etex}
\usepackage{savesym}
\usepackage[utf8]{inputenc}

\usepackage[T1]{fontenc}
\usepackage[small,bf]{caption}

% for plotting 
\usepackage{tikz}
\usepackage{pgfplots}
\usepackage{subfigure}
\usepackage{graphicx}
\usepackage{color}
\usepackage{xcolor}
    \definecolor{ListingsKeywordColor}{rgb}{0,0,0.4}
    \definecolor{ListingsIdentifierColor}{rgb}{0,0.5,0}
    \definecolor{ListingsCommentColor}{rgb}{0.4,0.4,0.4}
    \definecolor{ListingsStringColor}{rgb}{0.6000,0.3333,0.7333}%{0.8,0,0}
    \definecolor{ListingsRuleSepColor}{rgb}{0,0,0}
    \definecolor{ListingsEmphColor}{rgb}{0,0.6667,0.6667}
    \definecolor{ListingsBreakSymbolColor}{rgb}{0.780,0.082,0.522}
    \definecolor{LinkColor}{rgb}{0,0,0.5}
    \definecolor{UnitColor}{rgb}{0,0,0}
    \definecolor{MathsVectorColor}{rgb}{0,0,0}
    \definecolor{MathsMatrixColor}{rgb}{0,0,0}
    \definecolor{MyGreen}{HTML}{228B22}
%     \definecolor{TikzGreen}{HTML}{}
    \definecolor{MyBlue}{HTML}{0000FF}
    \colorlet{MatrixElementsLight}{gray!20!white}
    \colorlet{MatrixElementsDark}{gray!40}

% for listing
\usepackage{listings}
  \lstset{
    basicstyle=\scriptsize\ttfamily,
    tabsize=3,
    showtabs=false,
    showspaces=false,
    showstringspaces=false,
    tab=\rightarrowfill,
    keywordstyle=\color{ListingsKeywordColor},
    identifierstyle=\color{ListingsIdentifierColor},
    commentstyle=\color{ListingsCommentColor},
    stringstyle=\color{ListingsStringColor},
    emphstyle=\color{ListingsEmphColor}\bfseries\underbar,
    frame=none,
    rulesepcolor=\color{ListingsRuleSepColor},
    numbers=left,
    numberstyle=\tiny,
    numbersep=5pt,
    captionpos=top,
    frame=tb,
    firstnumber=1,
    stepnumber=1,
%     numberfirstline=false,
    breaklines=true,
    breakatwhitespace=true,
%     prebreak=\mbox{\,$\color{ListingsBreakSymbolColor}\mathbf{\hookleftarrow}$},
    mathescape=true,
    morekeywords={},
}

\usepackage{url}
\usepackage[final]{hyperref}
\hypersetup{
   % Farben fuer die Links
   colorlinks=false,         % Links erhalten Farben statt Kaeten
   urlcolor=LinkColor,    % \href{...}{...} external (URL)
   filecolor=LinkColor,  % \href{...} local file
   linkcolor=LinkColor,  %\ref{...} and \pageref{...}
   menucolor=LinkColor,
   citecolor=LinkColor,
   % Links
   raiselinks=true,       % calculate real height of the link
   breaklinks,              % Links berstehen Zeilenumbruch
   %backref=page,            % Backlinks im Literaturverzeichnis (section, slide, page, none)
   %pagebackref=true,        % Backlinks im Literaturverzeichnis mit Seitenangabe
   verbose,
   hyperindex=true,         % backlinkex index
   linktocpage=true,        % Inhaltsverzeichnis verlinkt Seiten
   hyperfootnotes=false,     % Keine Links auf Fussnoten
   % Bookmarks
   bookmarks=true,          % Erzeugung von Bookmarks fuer PDF-Viewer
   bookmarksopenlevel=1,    % Gliederungstiefe der Bookmarks
   bookmarksopen=false,      % Expandierte Untermenues in Bookmarks
   bookmarksnumbered=true,  % Nummerierung der Bookmarks
   bookmarkstype=toc,       % Art der Verzeichnisses
   % Anchors
   plainpages=false,        % Anchors even on plain pages ?
   pageanchor=true,         % Pages are linkable
   % PDF Informationen
   pdftitle={\reporttitle: \reportsubtitle. \reportsubsubtitle},             % Titel
   pdfauthor={Roman Iakymchuk},            % Autor
   pdfcreator={LaTeX, hyperref, KOMA-Script},
   pdfstartview=Fit,       % Dokument wird Fit Height geaefnet
   pdffitwindow=true,
   pdfpagemode=UseOutlines, % Bookmarks im Viewer anzeigen
   pdfpagelabels=true,      % set PDF page labels
}

\usepackage{cleveref}
\Crefname{algorithm}{Alg.}{Algs.}
\Crefname{figure}{Fig.}{Figs.}
\Crefname{table}{Tab.}{Tabs.}
\Crefname{equation}{Eq.}{Eqs.}
\Crefname{eqnarray}{Eq.}{Eqs.}

\usepackage{float}
\newfloat{algorithm}{htb}{lob}
\floatname{algorithm}{Algorithm}

\newlength{\figlength}
\setlength{\figlength}{7.6cm} %8.2cm
\newlength{\subfiglength}
\setlength{\subfiglength}{7.3cm}%7.8

\newcommand{\becomes}{:=}
\newcommand{\bytes}{b}
\newcommand{\kb}{Kb}
\newcommand{\mb}{Mb}
\newcommand{\ghz}{GHz}
\newcommand{\flop}{Flop}
\newcommand{\flops}{Flops}
\newcommand{\gflops}{G\flops}
\newcommand{\svn}{SVN}
\newcommand{\fxt}{FxT}
\newcommand{\gtg}{GTG}
\newcommand{\mpi}{MPI}


% Title Page
\title{
{\Huge\bf \reporttitle{}}\\[6mm]
{\LARGE\bf \reportsubtitle}\\[12mm]
{\Large\bf \reportsubsubtitle}}
% \author{Roman Iakymchuk}

\begin{document}
\maketitle

\tableofcontents

\chapter{License of \eztrace}
\eztrace{} is developed and distributed under the GNU General Public License.

This program is free software; you can redistribute it and/or modify it under the
terms of the GNU General Public License as published by the Free Software Foundation;
either version 2 of the License, or (at your option) any later version.

\eztrace{} is distributed in the hope that it will be useful, but WITHOUT ANY WARRANTY;
without even the implied warranty of MERCHANTABILITY or FITNESS FOR A PARTICULAR PURPOSE.
See the GNU General Public License for more details.

You should have received a copy of the GNU General Public License along with \eztrace{};
if not, write to the Free Software Foundation, Inc., 675 Mass Ave, Cambridge, MA 02139, USA.

\chapter{What is \eztrace{}?}
\eztrace{}~\cite{Aulagnon12EZTrace,TraRue11Ccgrid,Coulomb10periscope} is a tool that
aims at generating automatically execution traces from High-Performance Computing (HPC)
programs. It generates execution trace files that can be interpreted by visualization
tools such as ViTE~\cite{vite}.

\eztrace{} has been designed to provide a simple way to trace parallel applications.
This framework relies on plugins in order to offer a generic way to analyze programs;
depending on the application to analyze or on the point to focus on, several modules
can be loaded. \eztrace{} provides pre-defined plugins that give the ability to the user
to analyze applications that use \mpi{} libraries, OpenMP, or Pthreads. However, user-defined
plugins can also be loaded in order to analyze application functions or custom libraries.

\chapter{Installing \eztrace{}}
\section{Requirements}
In order to run \eztrace{}, the following software needs to be installed:
\begin{enumerate}
 \item autoconf 2.63;
 
 \item {[}Optional\,{]} Any \mpi{} implementation. Make sure your \mpi{} installation provides dynamic libraries.
\end{enumerate}

We use custom versions of the following software:
\begin{enumerate}
%  \item \fxt{} 0.2.10 or later. The latest release of \fxt{} is available here,
%        \url{http://download.savannah.gnu.org/releases/fkt/};
 \item \fxt{} 0.2.10 (\url{http://download.savannah.gnu.org/releases/fkt/});       
 \item \gtg{} 0.2 (\url{http://gforge.inria.fr/projects/gtg/}).
\end{enumerate}
Those are already included in \eztrace{}, but you can also provide your own versions.

\section{Getting \eztrace}
\begin{enumerate}
 \item You can get the latest stable release from the \eztrace{} website,
       \url{http://eztrace.gforge.inria.fr/};

 \item Current development version is available via \svn{}\\
       \texttt{svn checkout svn+ssh://scm.gforge.inria.fr/svn/eztrace}.
\end{enumerate}
After getting the latest development version (from \svn{}), you need to run\\
\hspace*{0.9cm}\texttt{./bootstrap}\\
and only then build the tool.

\section{Building \eztrace{}}
At first, you need to configure \eztrace{} by invoking the configure script:\\
\hspace*{0.9cm}\texttt{./configure -\,-prefix=<WHERE\_YOU\_INSTALL\_EZTRACE>}\\
Options to configure. You can pass several options to the configure script for
specifying where it should find the required libraries:\\
\begin{tabular}{@{\hskip .9cm}l}
\texttt{-\,-with-fxt=\$FXT\_ROOT} -- specify where \fxt{} is installed;\\
\texttt{-\,-with-gtg=\$GTG\_ROOT} -- specify where \gtg{} is installed;\\
\texttt{-\,-with-mpi=\$MPI\_ROOT} -- specify where \mpi{} is installed. The \texttt{mpi.h}\\
\hspace*{0.4cm}file must be located in the \texttt{\$MPI\_ROOT/include/directory};\\
\texttt{-\,-with-mpi-include=<PATH\_TO\_MPI.H>} -- specify the directory that\\
\hspace*{0.4cm}contains the \texttt{mpi.h} file.
\end{tabular}

Once \eztrace{} is configured, just run:\\
\hspace*{0.9cm}\texttt{make}\\
\hspace*{0.9cm}\texttt{make install}

\chapter{\eztrace{} in Details}
% {\bf Draw a picture where I present different parts of \eztrace{} such as core and modules.}

\eztrace{} uses a two-phases mechanism for analyzing performance. During the first phase
that occurs while the application is executed, functions are intercepted and events are
recorded. After the execution of the application, the post-mortem analysis phase is in
charge of interpreting the recorded events. This two-phase mechanism permits the library to
separate the recording of a function call from its interpretation. It thus allows the user
to interpret a function call event in different ways depending on the point he/she wants to
focus on. It also reduces the overhead of profiling a program; during the execution of the
application, the analysis tool should avoid performing time-consuming tasks such as computing
statistics or interpreting function calls.

During the execution of the application, \eztrace{} intercepts calls to the functions
specified by plugins and records events for each of them. Depending on the type of functions,
\eztrace{} uses two different mechanisms for interception. The functions defined in shared
libraries can be overridden using \texttt{LD\_PRELOAD}: When the \eztrace{} library is loaded,
it retrieves the addresses of the functions to instrument. When the application calls one of
these functions, the version implemented in \eztrace{} is called. This function records events
and calls the actual function. The \texttt{LD\_PRELOAD} mechanism cannot be used for functions
defined in the application since there is no symbol resolution. In that case, \eztrace{} uses
the DynInst~\cite{Buck00DynInst} tool for instrumenting the program on the fly. Using DynInst,
\eztrace{} modifies the program to record events at the beginning and/or at the end of each
function to instrument.

\eztrace{} is structurally divided into two parts: the \eztrace{} core and the \eztrace{} modules.

\section{\eztrace{} Core}
The \eztrace{} core is composed of several libraries such as the \fxt{}~\cite{Danjean05FxT} and the
Generic Trace Generator (\gtg{})~\cite{Coulomb10periscope,gtg} libraries. 

\subsection{\fxt{}}
For recording events, \eztrace{} relies on the \fxt{} library. Each process being instrumented
by \eztrace{} generates a trace file using \fxt{}. In order to keep the trace size as compact as
possible, \fxt{} records events in a binary format that contains only the minimum amount of
information: a timestamp, an event code and optional parameters.

\subsection{\gtg{}}
During the post-mortem analysis phase, \eztrace{} browses the recorded events and interprets them.
It can then generate statistics -- such as the length of messages, the duration of critical sections,
etc.-- or create a trace file for visualizing the application behavior. For generating trace files,
\eztrace{} relies on the \gtg{} library. \gtg{} provides an abstract interface for recording traces.
This permits the application to use a single interface for creating traces in multiple formats. Thus,
an application can generate PAJ\'E traces or OTF files without any modification.

% Once \eztrace{} is running along with the application, \fxt{} traces are generated. The second part of
% \eztrace{} is based on \gtg{}, and transforms the raw traces to real meaningful traces. First a meaning
% is added: for example, 42 represents an \texttt{MPI\_Send} request according to the \mpi{} plugin.
% The semantic can represent links, events, states, etc. The hierarchical structure of the generated trace is P AJÉ like, al-
% thought OTF traces can also be generated. The containers can have states (’ This
% thread is in this function’), notify events, or count relevant data (number of mes-
% sages, memory used, number of jobs, etc). This step is based on the plugins (plugins
% give different meaning to the symbols). Using the EZT RACE convert tool based on
% GTG, one can add meaning, define containers, and describe what is happening in a
% function.

Althought PAJ\'E and OTF are both traces format, they have some differences. Thus, adding a meaning
to a raw \fxt{} event is the critical part and the event must be interpreted in a way that is conformed
to the output format choosen by the user. Otherwise, the traces will not represent the data they should.

% \subsection{Opari}

\section{\eztrace{} Modules}
% The \eztrace{} modules
Since \eztrace{} uses the two-phases mechanism, plugins are organized in two parts, accordingly:
the description of the functions to instrument, and the interpretation of each function call.
During the execution of the application, the first part of the plugin is in charge of recording
calls to a set of functions. The second part of the plugin is in charge of adding semantic to the
trace. \eztrace{} provides plugins for major parallel programming libraries (MPI, OpenMP, PThread,
etc) but also allows user-defined plugins designed for custom libraries or applications. For exam-
ple, the PLASMA linear algebra library~\cite{Agullo09Plasma} is shipped with an EZTRACE plugin.

% \subsection{\tt mpi}
% \subsection{\tt openmp}
% \subsection{\tt pthread}
% \subsection{\tt stdio}
% \subsection{\tt memory}

\chapter{How to Use \eztrace}
\section{Selecting Functions to Instrument}
First, you should select functions in your application that you want to instrument.
For this, you can set the \texttt{EZTRACE\_TRACE} environment variable to the
list of plugins that should be used. For instance, set\\
\hspace*{0.9cm}\texttt{export EZTRACE\_TRACE="pthread mpi"}\\
if you want to instrument the PThread and \mpi{} functions.

You can get the list of available plugins using the \texttt{eztrace\_avail} command:\\
\hspace*{0.9cm}\texttt{eztrace\_avail}\\[1mm]
\begin{tabular}{@{\hskip 1.3cm}lll}
\tt 3 & \tt stdio & \tt Module for stdio functions (read, write,\\
 & & \tt select, poll, etc.)\\
\tt 2 & \tt pthread & \tt Module for PThread synchronization functions\\
 & & \tt (mutex, semaphore, spinlock, etc.)\\
\tt 6 & \tt papi & \tt Module for PAPI Performance counters\\
\tt 1 & \tt omp & \tt Module for OpenMP parallel regions\\
\tt 4 & \tt mpi & \tt Module for \mpi{} functions\\
\tt 5 & \tt memory & \tt Module for memory functions (malloc, free,\\
 & & \tt etc.)
\end{tabular}

You can get the list of selected modules with the \texttt{eztrace\_loaded} command:\\
\hspace*{0.9cm}\texttt{export EZTRACE\_TRACE="pthread mpi"}\\
\hspace*{0.9cm}\texttt{eztrace\_loaded}\\[1mm]
\begin{tabular}{@{\hskip 1.3cm}lll}
\tt 2 & \tt pthread & \tt Module for PThread synchronization functions\\
 & & \tt (mutex, semaphore, spinlock, etc.)\\
\tt 4 & \tt mpi & \tt Module for \mpi{} functions
\end{tabular}

\section{Generating Execution Traces}
Once the list of plugins is selected, you can run your application with \eztrace{}.
For example:\\
\hspace*{0.9cm}\texttt{eztrace ./my\_program my\_arg1 my\_arg2}\\
This command line executes your program and generates a trace file in the \texttt{/tmp}
directory (usually the file is named as \texttt{/tmp/<username>\_eztrace\_log\_\\rank\_<rank>}).

\section{Using \eztrace{} for \mpi{} Applications}
\eztrace{} needs to instrument each \mpi{} process. Thus, you can run the following 
command:\\
\hspace*{0.9cm}\texttt{mpirun -np nproc eztrace ./my\_program},\\
where {\tt npoc} is the number of \mpi{} processes. When your application ends. Each process
writes a file named\\
\hspace*{0.9cm}\texttt{/tmp/<username>\_eztrace\_log\_rank\_<rank>}.

\section{Changing the Output Directory}
\label{sec:changing_dir}
By default, each process saves its trace in the local \texttt{/tmp} directory. You can
change this by setting the \texttt{EZTRACE\_TRACE\_DIR} environment variable.

\section{Merging Execution Traces}
Once the execution traces are recorded, you can merge and convert them into a
file format that can be read by your visualization software:\\
\hspace*{0.9cm}\texttt{eztrace\_convert -o my\_paje.trace /tmp/<username>\_eztrace\_}\\
\hspace*{1.3cm}\texttt{log\_rank\_0 /tmp/<username>\_eztrace\_log\_rank\_1}
This converts the trace files into the Paje format. If \gtg{} is installed with OTF
support (this is enabled by default), you can choose to convert into the OTF
file format with the \texttt{-t OTF} option:\\
\hspace*{0.9cm}\texttt{eztrace\_convert -t OTF /tmp/<username>\_eztrace\_log\_rank\_0}\\
\hspace*{1.3cm}\texttt{/tmp/<username>\_eztrace\_log\_rank\_1}

\section{Filtering Events}
You can select the plugins to use for the conversion phase by using the \texttt{EZTRACE\_TRACE}
environment variable. For instance, if your traces contains \mpi{} and other events, then by setting
\texttt{EZTRACE\_TRACE} to \texttt{mpi} and calling \texttt{eztrace\_convert} you will receive
an output trace that contains only \mpi{} events.

\section{Computing Statistics}
Instead of creating a merged trace file, you can tell \eztrace{} to compute statistics on
the recorded traces:\\
\hspace*{0.9cm}\texttt{eztrace\_stats /tmp/<username>\_eztrace\_log\_rank\_0}\\
\hspace*{1.3cm}\texttt{/tmp/<username>\_eztrace\_log\_rank\_1}\\[1mm]
\begin{tabular}{@{\hskip 1.7cm}l}
\tt [...]\\
\tt PThread:\\
\tt --------------\\
\tt 6 locks acquired\\
\\
\tt MPI:\\
\tt -----\\
\tt 27 messages sent\\
\tt MPI\_RECV: 10 calls\\
\tt MPI\_BARRIER: 11 calls\\
\tt 163 events handled
\end{tabular}

\section{Defining Custom Plugins}
Since \eztrace{} works with plugins, you can create one and instrument the functions that
you want. An example of a custom plugin is available in the \texttt{example} directory.

Once your plugin is created, you should tell \eztrace{} where to find it. For this, just
set the \texttt{EZTRACE\_LIBRARY\_PATH} variable to the appropriate directory(-ies):\\
\hspace*{0.9cm}\texttt{export EZTRACE\_LIBRARY\_PATH=plugin1:plugin2}\\
\hspace*{0.9cm}\texttt{eztrace\_avail}\\[1mm]
\begin{tabular}{@{\hskip 1.3cm}lll}
\tt 3 & \tt stdio & \tt Module for stdio functions (read, write,\\
 & & \tt select, poll, etc.)\\
\tt 2 & \tt pthread & \tt Module for PThread synchronization func-\\
 & & \tt tions (mutex, semaphore, spinlock, etc.)\\
\tt 6 & \tt papi    & \tt Module for PAPI Performance counters\\
\tt 1 & \tt omp     & \tt Module for OpenMP parallel regions\\
\tt 4 & \tt mpi     & \tt Module for MPI functions\\
\tt 5 & \tt memory  & \tt Module for memory functions (malloc, free,\\
 & & \tt etc.)\\
\tt 99 & \tt plugin1 & \tt Example module for libplugin1\\
\tt 98 & \tt plugin2 & \tt Example module for the plugin2 library\\
\end{tabular}

\section{Generating Custom Plugins}
You can generate one plugin and instrument the functions you want.
In order to generate your plugin, you need to create a file containing:
\begin{enumerate}
 \item The name of the library you want to trace (\texttt{libNAME.so});
 \item {[}Optional{]} A brief description of the library;
 \item An ID to identify the module. 0 is reserved for \eztrace{} internal
  use. Thus, you can use any between 10 and ff;
 \item The prototype of the functions you want to instrument.
\end{enumerate}
As a result, your file should look as follow
\lstset{language=C, caption={\tt example.tpl}}
\begin{lstlisting}
BEGIN_MODULE 
NAME example_lib 
DESC "module for the example library"
ID 99
int example_do_event(int n)
double example_function1(double* array, int array_size)
END_MODULE
\end{lstlisting}

Now use \texttt{eztrace\_create\_plugin} to generate the plugin source
code:\\
\hspace*{0.9cm}\texttt{eztrace\_create\_plugin example.tpl}\\
\begin{tabular}{@{\hskip 1.3cm}l}
\tt New Module\\
\tt Module name: 'example\_lib'\\
\tt Module description: '"module for the example library"'\\
\tt Module id: '99'\\
\tt \hspace*{0.8cm} emulate record\_state for 'example\_do\_event'\\
\tt Function 'example\_do\_event' done\\
\tt \hspace*{0.8cm} emulate record\_state for 'example\_function1'\\
\tt Function 'example\_function1' done\\
\tt End of Module example\_lib
\end{tabular}\\
The source code is generated in the output directory. Just type:\\
\hspace*{0.9cm}\texttt{make}\\
Then, set the \texttt{EZTRACE\_LIBRARY\_PATH} to the appropriate directory. Now,
your custom plugin is ready to be used.

You can also determine (in the \texttt{example.tpl} file) the way a function is
depicted in the output trace. For instance,
\lstset{language=C, caption={}}
\begin{lstlisting}
int submit_job(int* array, int array_size)
BEGIN
 ADD_VAR("job counter", 1)
END
\end{lstlisting}
specifies that when the \texttt{submit\_job} function is called, the output
trace should increment the "job counter" variable. You can now track the value
of a variable.

The \texttt{test/module\_generator} directory contains several scripts that
demonstrate the various commands available.

\chapter{Environment Variables}
Here is a list of the environment variables that can be used for tuning \eztrace{}.
\begin{itemize}
 \item General-purpose variables:
  \begin{itemize}
   \item \texttt{EZTRACE\_TRACE\_DIR} specifies the directory in which trace files
    are stored (by default it is \texttt{/tmp});
   \item \texttt{EZTRACE\_LIBRARY\_PATH} specifies the directories in which \eztrace{}
    can find \eztrace{} modules (by default, it is none);
   \item \texttt{EZTRACE\_TRACE} specifies the list of \eztrace{} modules to load
    (by default, it is the list of all available modules);
   \item \texttt{EZTRACE\_BUFFER\_SIZE} specifies the size of the buffer in which 
    \eztrace{} stores events (by default, the size is 32\,MB);
   \item \texttt{EZTRACE\_FLUSH} specifies the behavior of \eztrace{} when the event
    buffer is full. If it is set to one, the buffer is flushed. This permits to record traces
    that are larger than \texttt{EZTRACE\_BUFFER\_SIZE}, but this may impact the
    application performance. Otherwise, if it is set to zero, which is a default option,
    any additional event will be recorded. The trace is, thus, truncated and there is no impact
    on performance.
  \end{itemize}
 \item Error-handling variables:
  \begin{itemize}
   \item \texttt{EZTRACE\_NO\_SIGNAL\_HANDLER} disables \eztrace{} signal handling (by default,
   it is zero).
  \end{itemize}

 \item Hardware counters-related variables:
  \begin{itemize}
   \item \texttt{EZTRACE\_PAPI\_COUNTERS} selects hardware events to trace using the PAPI
    library, e.g. export \texttt{EZTRACE\_PAPI\_COUNTERS="PAPI\_L3\\\_TCM PAPI\_FP\_INS"}.
    Please note that the list of supported events as well as the number of events,
    which can be traced simultaneously, vary depending on the processor type. This information
    can be retrieved using \texttt{papi\_avail} and the processor documentation.
  \end{itemize}
  
 \item MPI-related variables:
  \begin{itemize}
   \item \texttt{EZTRACE\_MPI\_DUMP\_MESSAGES} tells \eztrace{} to dump the list of messages
    into a file. You can then compute your own statistics on MPI messages.
  \end{itemize}    
\end{itemize}
    
\chapter{Known Limitations}
 \begin{itemize}

  \item If \eztrace{} is compiled with a particular \mpi{} implementation such as OpenMPI,
   it will not work if you run your application with another, e.g. MPICH2. So make sure
   your application uses the same \mpi{} implementation as \eztrace{}.
 \end{itemize}

\chapter{Frequently Asked Questions}
\begin{itemize}
 \item[\bf Q.] When I run my \mpi{} application with \eztrace{}, all the processes generate
  the \texttt{/tmp/<username>\_eztrace\_log\_rank\_1} file. What is going wrong?

 \item[\bf A.] This happens when \eztrace{} fails to intercept calls to \texttt{MPI\_Init}
  or \texttt{MPI\_Init\_thread}. There can be several reasons for that:
  \begin{itemize}
   \item The \eztrace{} \mpi{} module was not compiled. For intercepting calls to \mpi{} functions,
    you need the MPI module in your installation (look for the
    \texttt{EZTRACE\_ROOT/lib/libeztrace-autostart-\\mpi.so} file). If you do not see that file,
    it means that something went wrong during the configuration of \eztrace{}, so check for
    errors or warnings in the \texttt{config.log} file.
   \item You specified the list of modules to use and the \mpi{} module was not there. Check
    your \texttt{EZTRACE\_TRACE} variable or use \texttt{eztrace\_\\loaded}.
  \end{itemize}
  If you still experience problems, please contact the \eztrace{} development team and we will
  fix your problem.\\[1mm]
  
  \item[\bf Q.] What if I do not want to trace the whole application, but only a part of it?

  \item[\bf A.] Then, you can call \texttt{eztrace\_start()} and \texttt{eztrace\_stop()} specify 
   in the code which part to trace. You will need to add \texttt{\#include <eztrace.h>}
   and to link with \texttt{libeztrace}. Afterwards, you can run your application as usual,
   i.e. \texttt{./my\_program my\_arg1}.\\[1mm]
  
  \item[\bf Q.] I need to trace my program while using GDB, how can I do that?

  \item[\bf A.] Just add the \texttt{-d} option to \eztrace{} to enable GDB:\\
   \hspace*{0.9cm}\texttt{eztrace -d ./my\_program my\_arg1 my\_arg2}\\
   Please note that this should be applied only when a bug occurs while using \eztrace{}.\\[1mm]
  
  \item[\bf Q.] I want my trace to be saved in a specific directory, how can I do that?

  \item[\bf A.] Please take a look in \Cref{sec:changing_dir}.\\[1mm]
  
  \item[\bf Q.] What if I do not care about OpenMP and I only want to see \mpi{} communication?

  \item[\bf A.] You can set \texttt{EZTRACE\_TRACE} to the list of modules you want to activate.
   By default, all the available modules are enabled, but you can tell \eztrace{} to trace only
   \mpi{}, OpenMP, or both \mpi{} and OpenMPI functions:\\[1mm]
   \begin{tabular}{@{\hskip 0.9cm}l}
    \tt export EZTRACE\_TRACE=mpi\\
    \tt export EZTRACE\_TRACE=omp\\
    \tt export EZTRACE\_TRACE="omp mpi"
   \end{tabular}\\[1mm]
  
  \item[\bf Q.] Can \eztrace{} generate an OTF trace file so that I can visualize it with 
   Vampir?

  \item[\bf A.] Yes, since \eztrace{} relies on GTG for writing output traces, it can generate
   OTF trace files. When converting the trace with \texttt{eztrace\_con\\vert}, just add the
   \texttt{-t OTF} option:\\
   \hspace*{0.9cm}\texttt{eztrace\_convert -t OTF /tmp/<username>\_eztrace\_log\_\\}
   \hspace*{1.3cm}\texttt{rank\_0 /tmp/<username>\_eztrace\_log\_rank\_1}
\end{itemize}

\chapter{Troubleshooting}
If you encounter a bug or want some explanation about \eztrace{}, please contact and ask our
development team:
\begin{itemize}
 \item On the development mailing list, \url{https://gforge.inria.fr/mail/?group_id=2774};

 \item On our IRC channel:
  \begin{itemize}
   \item Server: \texttt{chat.freenode.net}
   \item Channel: \texttt{\#eztrace}
  \end{itemize}
\end{itemize}

\bibliographystyle{plain}
\cleardoublepage
\phantomsection
\addcontentsline{toc}{chapter}{Bibliography}
\small
\bibliography{references}
\normalsize

\end{document}
